\chapter{Nachrichten und Signale}
\begin{description}
\item[Signal] elementare Informationsträger mit folgenden Eigenschaften:
\begin{itemize}
\item Gebunden an Übertragungsmedien
\item Unterliegt einer zeitlichen Veränderung (Zeitfunktion)
\end{itemize}
\item[Nachricht] Sammlung von Signalen
\end{description}

\section{Übertragung von Signalen}











\section{Analoge und digitale Signale}

Ein Signal nimmt zu einem bestimmten Zeitpunkt $t$ einen bestimmten Wert $w$ an: $w(t)$

\subsection{Kontinuierliche oder diskrete Signale}

Ein Signal kann sowohl im Wert als auch in der Zeit kontinuierlich oder diskret sein. 

\begin{multicols}{2}
\minisec{Zeit- und wertkontinuierlich:}

\begin{pspicture}(-0.5,-0.5)(5,2) 
\psaxes[linewidth=1pt,labels=none,ticks=none]{->}(0,0)(0,0)(5,2) 
\uput{1ex}[180](0,1.9){$w$} 
\uput{1ex}[270](4.9,0){$t$}

\psplot[plotstyle=curve,linecolor=red]{0.01}{5}{x 2 mul SIN 2 div 1 add}

\end{pspicture}

\minisec{Wertkontinuierlich und zeitdiskret:}

\begin{pspicture}(-0.5,-0.5)(5,2) 
\psaxes[linewidth=1pt,labels=none,ticks=x]{->}(0,0)(0,0)(5,2) 
\uput{1ex}[180](0,1.9){$w$} 
\uput{1ex}[270](4.9,0){$t$}

\psline[linecolor=red](0.5,0)(0.5,1.7)
\psline[linecolor=red](1,0)(1,0.9)
\psline[linecolor=red](1.5,0)(1.5,0.8)
\psline[linecolor=red](2,0)(2,1.2)
\psline[linecolor=red](2.5,0)(2.5,1.3)
\psline[linecolor=red](3,0)(3,1.5)
\psline[linecolor=red](3.5,0)(3.5,1.6)
\psline[linecolor=red](4,0)(4,1.0)

\end{pspicture}

\minisec{Wertdiskret und zeitkontinuierlich:}

\begin{pspicture}(-0.5,-0.5)(5,2) 
\psaxes[linewidth=1pt,labels=none,ticks=x]{->}(0,0)(0,0)(5,2) 
\uput{1ex}[180](0,1.9){$w$} 
\uput{1ex}[270](4.9,0){$t$}

\psline[linecolor=red](0,1)(1,1)
\psline[linecolor=red](1,1)(1,0)
\psline[linecolor=red](1,0)(2,0)
\psline[linecolor=red](2,0)(2,1)
\psline[linecolor=red](2,1)(4,1)
\psline[linecolor=red](4,1)(4,0)
\psline[linecolor=red](4,0)(5,0)

\end{pspicture}

\minisec{Wert- und zeitdiskret:}

\begin{pspicture}(-0.5,-0.5)(5,2) 
\psaxes[linewidth=1pt,labels=none,ticks=none]{->}(0,0)(0,0)(5,2) 
\uput{1ex}[180](0,1.9){$w$} 
\uput{1ex}[270](4.9,0){$t$}

\qdisk(0.5,1.7){2pt} 
\qdisk(1,0.9){2pt} 
\qdisk(1.5,0.8){2pt} 
\qdisk(2,1.2){2pt} 
\qdisk(2.5,1.3){2pt} 
\qdisk(3,1.5){2pt} 
\qdisk(3.5,1.6){2pt} 
\qdisk(4,1.0){2pt} 

\end{pspicture}
\end{multicols}

\subsection{Quantisierung von analogen Signalen}

Ist die Zuordnung von Zeitintervallen und Klassifizierung in diskrete Werte.

\begin{pspicture}(-0.5,-0.5)(10,2) 
\psaxes[linewidth=1pt,labels=none,ticks=none]{->}(0,0)(0,0)(5,2) 
\uput{1ex}[180](0,1.9){$w$} 
\uput{1ex}[270](4.9,0){$t$}

\psplot[plotstyle=curve]{0.01}{5}{x 2 mul SIN 2 div x 4 div 1 add mul 1 add}
\psline[linecolor=red]{->}(0.5,0)(0.5,1.5)
\psline[linecolor=red,linestyle=dashed](0.5,1.5)(1,1.5)
\psline[linecolor=red]{->}(1.0,0)(1.0,1.6)
\psline[linecolor=red,linestyle=dashed](1,1.6)(1.5,1.6)
\psline[linecolor=red,linestyle=dashed](1.5,1.6)(1.5,1.1)
\psline[linecolor=red]{->}(1.5,0)(1.5,1.1)
\psline[linecolor=red,linestyle=dashed](1.5,1.1)(2,1.1)
\psline[linecolor=red,linestyle=dashed](2,1.1)(2,0.4)
\psline[linecolor=red]{->}(2.0,0)(2.0,0.4)
\psline[linecolor=red,linestyle=dashed](2,0.4)(2.5,0.4)
\psline[linecolor=red,linestyle=dashed](2.5,0.4)(2.5,0.2)
\psline[linecolor=red]{->}(2.5,0)(2.5,0.2)
\psline[linecolor=red,linestyle=dashed](2.5,0.2)(3,0.2)
\psline[linecolor=red]{->}(3.0,0)(3.0,0.7)
\psline[linecolor=red,linestyle=dashed](3,0.7)(3.5,0.7)
\psline[linecolor=red]{->}(3.5,0)(3.5,1.6)
\psline[linecolor=red,linestyle=dashed](3.5,1.6)(4,1.6)
\psline[linecolor=red]{->}(4.0,0)(4.0,2.0)
\psline[linecolor=red,linestyle=dashed](4,2.0)(4.5,2.0)
\psline[linecolor=red,linestyle=dashed](4.5,2.0)(4.5,1.4)
\psline[linecolor=red]{->}(4.5,0)(4.5,1.4)

\psline[linecolor=red]{->}(5.6,1.3)(6,1.3)
\uput{1ex}[0](6,1.3){Abtastung} 
\psline[linecolor=red,linestyle=dashed](5.6,0.8)(6,0.8)
\uput{1ex}[0](6,0.8){Klasifizierung} 

\end{pspicture}
\definecolor{darkgreen}{rgb}{0,.6,0}
\begin{description}
\item[Digitales Signal] endlich viele Werte
\begin{description}
\item[Binär] zwei Werte
\item[Ternär] drei Werte
\end{description}
\item[analoges Signal] unendlich viele Werte
\end{description}

\section{Schritt- und Übertragungsgeschwindigkeit}

\paragraph{Einheitsschritt} $T_s$ Zeitdauer für die Übertragung eines Signals:

\begin{pspicture}(-0.5,-1)(5,1.5) 
\psaxes[linewidth=1pt,labels=none,ticks=x]{->}(0,0)(0,-1)(5,1) 
\uput{1ex}[270](4.9,0){$t$}

\psline[linecolor=darkgreen]{-}(0,0.5)(1,0.5)
\psline[linecolor=darkgreen]{-}(1,0.5)(1,-0.5)
\psline[linecolor=darkgreen]{-}(1,-0.5)(2,-0.5)
\psline[linecolor=darkgreen]{-}(2,-0.5)(2,0.5)
\psline[linecolor=darkgreen]{-}(2,0.5)(4,0.5)
\psline[linecolor=darkgreen,linestyle=dashed]{-}(3,-0.5)(3,0.5)
\psline[linecolor=darkgreen]{-}(4,0.5)(4,-0.5)
\psline[linecolor=darkgreen]{-}(4,-0.5)(4.7,-0.5)
\psline[linecolor=red,linewidth=1.2pt]{[-]}(0,0)(1,0)
\uput{1ex}[270](0.5,0){\color{red}$T_s$}

\end{pspicture}

\paragraph{Schrittgeschwindigkeit} $S=\frac{1}{T_s}$ Maßeinheit: Baud% (1 Baud = 1 Signal je Sekunde)

\paragraph{Signal} Kennzustand für die Dauer eines Einheitsschrittes

\begin{pspicture}(-0.5,-1)(5,1.5) 
\psaxes[linewidth=1pt,labels=none,ticks=x]{->}(0,0)(0,-1)(5,1) 
\uput{1ex}[270](4.9,0){$t$}

\psplot[plotstyle=curve,linecolor=red]{0.01}{1}{x PI 4 mul mul SIN 2 div}
\psplot[plotstyle=curve,linecolor=red]{1}{2}{x PI 2 mul mul SIN 2 div}
\psplot[plotstyle=curve,linecolor=red]{2}{4}{x 4 PI mul mul SIN 2 div}
\psplot[plotstyle=curve,linecolor=red]{4}{4.8}{x PI 2 mul mul SIN 2 div}

\uput{1ex}[90](0.5,0.5){\color{red}$1$}
\uput{1ex}[90](1.5,0.5){\color{red}$0$}
\uput{1ex}[90](2.5,0.5){\color{red}$1$}
\uput{1ex}[90](3.5,0.5){\color{red}$1$}
\uput{1ex}[90](4.5,0.5){\color{red}$1$}

\end{pspicture}

\subsection{Übertragungsgeschwindigkeit}

$U = bit/s$

Wichtig: Bei nicht-binärer Übertragung U != S

Beispiel: Übertragung mit 4 Kennzuständen

\begin{pspicture}(-2,-1)(5,2) 
\psaxes[linewidth=1pt,labels=none,ticks=x]{->}(0,0)(0,-1)(5,1.5) 
\uput{1ex}[270](4.9,0){$t$}

\psline[linecolor=darkgreen]{-}(0,0)(1,0)
\psline[linecolor=darkgreen]{-}(1,0)(1,0.5)
\psline[linecolor=darkgreen]{-}(1,0.5)(2,0.5)
\psline[linecolor=darkgreen]{-}(2,0.5)(2,1)
\psline[linecolor=darkgreen]{-}(2,1)(3,1)
\psline[linecolor=darkgreen]{-}(3,1)(3,-0.5)
\psline[linecolor=darkgreen]{-}(3,-0.5)(4,-0.5)

\uput{1ex}[180](0,0){\color{red}$00_{2} = 0$}
\psline[linecolor=red,linewidth=0.5pt,linestyle=dashed]{-}(0,0)(4.9,0)
\uput{1ex}[180](0,0.5){\color{red}$01_{2} = 1$}
\psline[linecolor=red,linewidth=0.5pt,linestyle=dashed]{-}(0,0.5)(4.9,0.5)

\uput{1ex}[180](0,1){\color{red}$10_{2} = 2$}
\psline[linecolor=red,linewidth=0.5pt,linestyle=dashed]{-}(0,1)(4.9,1)

\uput{1ex}[180](0,-0.5){\color{red}$11_{2} = 3$}
\psline[linecolor=red,linewidth=0.5pt,linestyle=dashed]{-}(0,-0.5)(4.9,-0.5)


\end{pspicture}

In einem Einheitsschritt können 2 Bit übertragen werden (S=1; U=2)

\section{Übertragungsmedien}

\begin{tabular}{p{0.25\textwidth}|p{0.35\textwidth}|p{0.35\textwidth}}

Art & Vorteile & Nachteile\\
\hline
\hline
elektrischer Strom (Metallkabel) & Kostengünstig, (stabil) & Große Entfernungen, bauliche Eingriffe \\
\hline
Radiowellen, Infrarot, Mikrowelle (Luft, Vakuum) & Kein Medium nötig & Langsam, störanfällig, u.U. Störquelle, Energiebedarf\\
\hline
Lich (LWL) & Extrem schnell, hoher Durchsatz & Teuer, reparaturunfreundlich

\end{tabular}

\subsection{Bemerkungen}

\minisec{Elektrischer Strom}
Material: Kupfer — bestes Preis/Leistungsverhältnis
Aufbau: 
Twised Pairs - Verdrillte Kabelpaare (Verringerung von Interferenzen)
Koaxialkabel

\minisec{Elektromagnetische Wellen}
Bedingung: Sender und Empfänger (Realisationen) müssen quasioptisch verbunden sein. Zusätzliche Empfangseinrichtung notwendig. 
Problem: Abschirmung der übertragenen Informationen

\minisec{Licht}
Basis: Bessere Kodierungsmöglichkeit (Multiplex) des Lichtes
Vorteil: keine elektromagnetischen Störungen

