\chapter{Nachrichten und Signale}
\begin{description}
\item[Signal] elementare Informationsträger mit folgenden Eigenschaften:
\begin{itemize}
\item Gebunden an Übertragungsmedien
\item Unterliegt einer zeitlichen Veränderung (Zeitfunktion)
\end{itemize}
\item[Nachricht] Sammlung von Signalen
\end{description}

\section{Übertragung von Signalen}











\section{Analoge und digitale Signale}

Ein Signal nimmt zu einem bestimmten Zeitpunkt $t$ einen bestimmten Wert $w$ an: $w(t)$

\subsection{Kontinuierliche oder diskrete Signale}

Ein Signal kann sowohl im Wert als auch in der Zeit kontinuierlich oder diskret sein. 

\paragraph{Zeit- und wertkontinuierlich}

\begin{pspicture}(-0.5,-0.5)(5,2) 
\psaxes[linewidth=1pt,labels=none,ticks=none]{->}(0,0)(0,0)(5,2) 
\uput{1ex}[180](0,1.9){$w$} 
\uput{1ex}[270](4.9,0){$t$}

\psplot[plotstyle=curve]{0.01}{5}{x 2 mul SIN 2 div 1 add}

\uput{1ex}[270](2,1.5){$w(t)$}

\end{pspicture}

\paragraph{Wertkontinuierlich und zeitdiskret}

\begin{pspicture}(-0.5,-0.5)(5,2) 
\psaxes[linewidth=1pt,labels=none,ticks=x]{->}(0,0)(0,0)(5,2) 
\uput{1ex}[180](0,1.9){$w$} 
\uput{1ex}[270](4.9,0){$t$}

\psline[linecolor=red](0.5,0)(0.5,1.7)
\psline[linecolor=red](1,0)(1,0.9)
\psline[linecolor=red](1.5,0)(1.5,0.8)
\psline[linecolor=red](2,0)(2,1.2)
\psline[linecolor=red](2.5,0)(2.5,1.3)
\psline[linecolor=red](3,0)(3,1.5)
\psline[linecolor=red](3.5,0)(3.5,1.6)
\psline[linecolor=red](4,0)(4,1.0)

\end{pspicture}

\paragraph{Wertdiskret und zeitkontinuierlich}

\begin{pspicture}(-0.5,-0.5)(5,2) 
\psaxes[linewidth=1pt,labels=none,ticks=x]{->}(0,0)(0,0)(5,2) 
\uput{1ex}[180](0,1.9){$w$} 
\uput{1ex}[270](4.9,0){$t$}

\psline[linecolor=red](0,1)(1,1)
\psline[linecolor=red](1,1)(1,0)
\psline[linecolor=red](1,0)(2,0)
\psline[linecolor=red](2,0)(2,1)
\psline[linecolor=red](2,1)(4,1)
\psline[linecolor=red](4,1)(4,0)
\psline[linecolor=red](4,0)(5,0)

\end{pspicture}

\paragraph{Wert- und zeitdiskret}

\begin{pspicture}(-0.5,-0.5)(5,2) 
\psaxes[linewidth=1pt,labels=none,ticks=none]{->}(0,0)(0,0)(5,2) 
\uput{1ex}[180](0,1.9){$w$} 
\uput{1ex}[270](4.9,0){$t$}

\qdisk(0.5,1.7){2pt} 
\qdisk(1,0.9){2pt} 
\qdisk(1.5,0.8){2pt} 
\qdisk(2,1.2){2pt} 
\qdisk(2.5,1.3){2pt} 
\qdisk(3,1.5){2pt} 
\qdisk(3.5,1.6){2pt} 
\qdisk(4,1.0){2pt} 

\end{pspicture}

\subsection{Quantisierung von analogen Signalen}

Ist die Zuordnung von Zeitintervallen und Klassifizierung in diskrete Werte. (Dia Info3)

\begin{description}
\item[Digitales Signal] endlich viele Werte
\begin{description}
\item[Binär] zwei Werte
\item[Ternär] drei Werte
\end{description}
\item[analoges Signal] unendlich viele Werte
\end{description}