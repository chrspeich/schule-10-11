\chapter{Computernetze}

\section{Netztopologien}

\paragraph{Bustopologie}

\begin{description}
\item[Vorteile]
\begin{itemize}
\item Einfache Realisierung
\end{itemize}
\item[Nachteile]
\begin{itemize}
\item Alle Teilnehmer müssen sich die max. Datengeschwindigkeit teilen
\item Bei defektem Bussegment Erreichbarkeit einiger Rechner nicht gesichert
\item Terminator erforderlich (Abschlusswiderstand)
\end{itemize}
\item[Beispiele] ISDN, ThinWire-Koaxialkabel
\end{description}

\section{Sterntopologie}

\begin{description}
\item[Vorteile]
\begin{itemize}
\item Unabhängigkeit der einzelnen Computer
\end{itemize}
\item[Nachteile]
\begin{itemize}
\item Aufwendigkeit
\item Bei Defekt des zentralen Computers/Koppelelements kein Betrieb möglich
\end{itemize}
\item[Beispiele] Ethernet
\end{description}

\section{Ringtopologie}

\begin{description}
\item[Vorteile]
\begin{itemize}
\item Betrieb kann bei Ausfall eines Ringsegments aufrecht erhalten werden
\end{itemize}
\item[Nachteile]
\begin{itemize}
\item Kommunikation immer nur zwischen zwei Computern
\end{itemize}
\item[Beispiele] Token-Ring, FDDI
\end{description}

\section{Maschentopologie}

\begin{description}
\item[Vorteile]
\begin{itemize}
\item Sehr effiziente Kommunikation zwischen den Computern möglich
\end{itemize}
\item[Nachteile]
\begin{itemize}
\item Extrem aufwendig
\end{itemize}
\item[Beispiele]
\end{description}


\section{Baumtopologie}

\begin{description}
\item[Vorteile]
\begin{itemize}
\item Aufgabenstruktur kann abgebildet werden
\end{itemize}
\item[Nachteile]
\begin{itemize}
\item Oft nur temporäre Lösung
\end{itemize}
\item[Beispiele]
\end{description}

\section{Zugriffsverfahren}

\subsection{Vorüberlegung Signallaufzeit}

Bei jedem Kabel existiert eine Konstante k, mit welcher die Ausbreitungsgeschwindigkeit  elektrischer Signale beschrieben wird. (Vakuum: 1.0c, Koaxialkabel: 0.77c, TwistedPair 0.66c)

Zwischen Station A und B soll kommuniziert werden. Station A sendet zum Zeitpunkt t eine Nachricht. Station B sendet eine Nachricht zum Zeitpunkt t+T-dt. Siehe Zeichnung

Für die Information das eine Nachricht erfolgreich von Station A zur Station B übermittet worden ist wird die doppelte Signallaufzeit 2T mit T =d/k*c, wobei d - Räumliche Entfernung, c - Lichtgeschwindigkeit, k - Materialkonstante.

Signallaufzeit spielt bei Zugriffsverfahren eine besondere Rolle. 

\subsection{Deterministische Zugriffsverfahren}

\begin{description}
\item[Problem] Wollen mehrere Computerpaare gleichzeitig im Netz (z.b. Ring) kommunizieren, kann es zur Kollision kommen. Die Pegel der Nachrichten können sich überlagern oder auslöschen, wodurch die Signalfolge unbrauchbar wird. 
\item[Ausweg] Einrichtung eines Berechtigungszeichens zum Senden (Token): nur der zeitweilige Besitzer des Zeichens darf senden. Nach Übermittlung wird das Token-Element weitergegeben.
\end{description}

\subsection{Stochastische Zugriffsverfahren}

Unter Verwendung von Zufallsinformationen wird der Zugriff einzelner Computer auf Netzwerkverbindungen reguliert. 

\paragraph{Ursprung: ALOHA-Verfahren (1970 Uni Hawaii)}
Funknetz wird nach dem Prinzip betrieben: Jeder darf zu jedem Zeitpunkt senden; Bestätigung über separaten Kanal. 

Bei Bestätigung -> Weitersenden

Bei ausbleibender Bestätigung (Kollision) ->  zufällige Wahl eines neuen neuen Sendezeitpunktes

=> Datenstau wird abgebaut

Optimale Sendeverhältnisse, wenn 18% der möglichen Sendezeit genutzt werden. 

\subparagraph{Modifikation} jeder darf nur zu Beginn eines festgelegten Zeitintervalls versuchen zu senden.

\paragraph{CSMA/CD-Verfahren}
(Carrier Sense Multiple Access / Collision Detect)

\begin{itemize}
\item Hineinhören in die Leitung
\item Falls freie Leitung: Sendebeginn
\item Mithören während der gesamten Sendung
\item Falls Kollision: Sendende Stationen produzieren JAM-Signal.
\item Übertragung der Datenpakete wird abgebrochen
\item Sendewillige Stationen warten einen zufällig bestimmten Zeitraum und versuchen es erneut
\end{itemize}

\paragraph{JAM-Signal} 32bit-Folge: 101010… 

\paragraph{Wichtig} Für die Erkennung von Kollisionen muss die Dauer Übertragung eines Datenblocks mindestens 2T betragen. 

\paragraph{Beispiel} 10Base-Ethernet => 10Mbit/s, 2,5km Stationsentfernung, 64byte Paketgröße => 512bit. t = 51,2 Mikro Sekunden.
10Base -> Koaxialkabel => k=0,77
Zurückgelegter Weg der Nachrichtenspitze im Koaxialkabel: $0,77 * 3*10^5 * 51,2*10^-3 = 11827m$

2T = 2,16e-4 Sekunden

\paragraph{Merke} 
\begin{itemize}
\item Der Konfliktparameter k = Maximale Signallaufzeit /  Nachrichtenübertraungszeit muss kleiner 1 sein
\item Bei den üblichen Netzstandards ergibt sich für die größte zulässige Netzlänge und die kleinste zulässige Paketgröße k ungefähr 0,21
\end{itemize}

$ Wartezeit = ZZ \cdot T $

ZZ Zufallszahl aus $0<ZZ<2^n$

N - Anzahl der Wiederholungen (max 10)
