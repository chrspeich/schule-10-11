\chapter{Datenübertragung}

\section{Spannungs-Zeit-Diagramm}

Bestand einer Nachrichtenübertragung:

\begin{enumerate}
\item Start- und Stoppsignal
\item eigentliche Nachricht
\item Paritätssignal (Sicherheitsinformation)
\end{enumerate}

Betrachtung zunächst an der Übertragung binärer Signale (0=+15, 1=-15)
Grundlage: RS 232 - Standard der seriellen Datenübertragung
\begin{itemize}
\item Kufperkabel
\item Spannung zwischen 3 und 15 Volt (0 Volt nicht zu gelassen)
\item Ruhezustand: -15V
\item Startbit: +15V
\item Stoppbit: -15V (1/1,5/2 Bit)
\item Paritätsbit
\end{itemize}
Parität: gerade oder ungerade bedeutet, dass die Anzahl der übertragenen 1-Informationen durch das Paritätsbit zu einer geraden/ungeraden Zahl ergänzt wird.

Bei binärer Information entspricht die Baudzahl der Zahl der übertragenen Bits(einschließlich Startt-/Stopp-/Paritätsbit). 

Übliche Baudzahlen: 110, 300, 600, 1200, …, 115200
Übliche Anzahl der Datenbits: 4, 5, 6, 7, 8

% -> Abbildung Info1 mit Aufgaben

\paragraph{Bemerkungen:}
\begin{enumerate}
\item Spannungs-Zeit-Diagramme werden üblicherweise für die serielle Datenübertragung angefertigt (Bitfolge nacheinander).

Zur Erhöhung der Übertragungsgeschwindigkeit kann die die parallele Datenübertragung dienen (mehrere Bit werden gleichzeitig übertragen — teurer).
\item Zur Erhöhung der Übertragungsgeschwindigkeit kann auch eine Übertragung mit mehr als 2 Kennzuständen verwendet werden. 
\begin{itemize}
\item 4 Kennzustände: 1Baud = 2bit
\item 8 Kennzustände: 1Baud = 3bit
\end{itemize}
\item Erhöhung der Übertragungsgeschwindigkeit auch durch Verkürzung des Einheitsschrittes möglich (begrenzt)
\end{enumerate}

\section{Modulationsarten}

\subsection{Modulierung als Rechtecksignale}
Grundlage: Fourier-Synthese

$$ f(t) = \sum\limits_{i=1}^\infty a_i \cdot sin(i \cdot t) + b_i \cdot cos(i \cdot t) $$


\subsection{Frequenzmodulation}

\subsection{Amplitudenmodulation}

\subsection{Phasenumtastung}

Binäre Werte werden einer Phasenverschiebung zugeordnet:
\begin{description}
\item[0] $= 0 \cdot \lambda $
\item[1] $= \frac{1}{4} \cdot \lambda $
\end{description}

\paragraph{Erweiterung} Quad Phase Shift Keying
\begin{description}
\item[00] $= 0 \cdot \lambda $
\item[01] $= \frac{1}{4} \cdot \lambda $
\item[10] $= \frac{1}{2} \cdot \lambda $
\item[11] $= \frac{3}{4} \cdot \lambda $
\end{description}

\paragraph{Bemerkung} Mit Hilfe eines Modems können im Frequenzbereich des der Telefonie (300-3400 Hz) digitale in analoge Signale umgewandelt werden und umgekehrt. 

\section{Gleichzeitige Übertragung mehrerer Nachrichten (Multiplexverfahren)}

\begin{center}
\begin{pspicture}(0,0)(10,2)

\psline{->}(0,0.7)(1,0.7)
\psline{->}(0,0.9)(1,0.9)
\psline{->}(0,1.1)(1,1.1)
\psline{->}(0,1.3)(1,1.3)

\psframe(1,0.4)(4,1.6)
\rput(2.5,1){Multiplexer}

\psline{->}(4,1)(6,1)

\psframe(6,0.4)(9,1.6)
\rput(7.5,1){Demultiplexer}

\psline{->}(9,0.7)(10,0.7)
\psline{->}(9,0.9)(10,0.9)
\psline{->}(9,1.1)(10,1.1)
\psline{->}(9,1.3)(10,1.3)

\end{pspicture}
\end{center}

\begin{description}
\item[Zeitmultiplexing] Unterteilung der Zeit in kleine Teilintervalle. Periodisch wird jeder zu übertragenden Nachricht ein Teilintervall zugeordnet. 
\item[Frequenzmultiplexing] Aufprägung mehrerer Nachrichten in unterschiedlichen Frequenzbereichen auf eine Grundfrequenz und Zerlegung des gesamten Frequenzspektrums beim Empfänger.

Oversampling: Abtasten mit dem mehrfachen der Grundfrequenz. 
\end{description}

\section{Hardwarebandbreite und Bitübertragung}

Hardware kann Spannung nicht in beliebig kurzen Intervallen wechseln. Das Rechteckdiagramm ist eine Idealisierung. Deswegen Begrenzung der höchsten Geschwindigkeit eines Signals.

%->Abbildung

\textbf{Bandbreite $B$ ist eine Eigenschaft des Übertragungsmediums.} (Maßeinheit: Herz)

\section{Nyquist-Theorem}

Stellt Zusammenhang zwischen der Bandbreite des Übertragungssystems und der theoretischen Höchstgeschwindigkeit $D$ dar:

$$ D = 2 \cdot B \cdot log_2 k $$

$k$ \ldots Anzahl der Kennzustände

$D$ \ldots theoretische Höchstgeschnwindigkeit in $\frac{Bits}{Sekunde}$

\section{Handshake und Betriebsarten}

Handshake bezeichnet den Austausch von Informationen über die Sende- und Empfangsbereitschaft zusätzlich zur eigentlichen Nachrichtenübertragung.

\begin{description}
\item[Polling] Gegenstelle wird zyklisch auf Sendebereitschaft abgefragt

Vorteil: einfache Realisierung

Nachteil: keine Sendung zum beliebigem Zeitpunkt
\item[Software-Handshakeing] Mitteilung des Senders an den Empfänger, dass Sendebereitschaft vorliegt(XON) oder nicht (XOFF)

Nachteil: 3 Leitungen (2 Nachrichten + 1 Signal)

Vorteil: große Entfernungen "noch machbar". (bidirektional)
\item[Hardware-Handshakeing] Sendung der Informationen: 0(Ready) durch Empfänger an DTR (Data Terminal Ready). Oder Sendung von 1 (Busy=Empfangsbuffer voll). 

Vorteil: Empfangszustand jedes Teilnehmers ist jederzeit verfügbar. 

Nachteil: Zur Realisierung werden 7 Datenleitungen benötigt. 
\end{description}

\section{Ethernet}

Ethernet ist grundlegende Norm für die Datenübertragung. 

\paragraph{Bestandteile}
\begin{itemize}\marginpar{Netzmafia Ethernet Typen}
\item Medium: Koaxialkabel, Twisted Pair (Steckverbindungen)
\item Übertragungsrate: z.B. 10Mbit/s
\item Maximale Ausdehnung des Netzes: 2500m(Signallaufzeit/-stärke)
\item Zugriffsverfahren: deterministische und stochastische Zugriffsverfahren
\item Protokoll: Paketinformationen, Adressen
\item Beziehungen zum OSI-Schichtenmodell
\end{itemize}

%\subsection{CRC}
%TODO

\section{Paket}

\paragraph{Ergänzungen}
Übertragendes Paket besteht aus Frame und Datenblock. %(Skizze)
Frame beinhaltet im Header die Synchronisationsinformation.

\begin{description}
\item[Asynchrone Übertragung] Sender und Empfänger müssen ihre Übertragung zeitlich nicht verabreden (Notwendigkeit von Signalen, die Beginn und Ende der Übertragung eines Zeichens kennzeichnen)
\item[Synchrone Übertragung] Empfangstakt wird auf Sendetakt abgestimmt
Im Header des Pakets (durch Übertragung eines Synchronisationszeichens, etwa ASCII-SYN). Nach Empfang der Zeichenfolge "rastet" Empfänger ein.
\end{description}

Am Ende des Paketes werden Prüfsummen- bzw. CRC-Informationen über den Datenblock gesendet (Sicherung) und der Block wird abgeschlossen. 

\section{Betriebsarten}
\begin{description}
\item[Simplex] Kommunikation in einer Richtung
\item[Halbduplex] Wechselseitige Nutzung der Übertragungsmedien zum Senden und Empfangen
\item[Duplex] Beide Teilnehmer sind über den gesamten Zeitraum Sender und Empfänger zugleich.
\end{description}


